Evolutionary Algorithms (EAs) are in general population-based optimization algorithms, adopting a higher-level of abstraction of an evolutionary scheme found in nature.
The classical Simple Genetic Algorithm (SGA) involves individuals, phenotypes, genotypes, fitness function, selection, recombination, and mutation.
Every individual is a candidate in solution space, with their genes being the parameters of the solution, and the phenotypes being the resulting characteristics of the solution, when in contact with the environment.
The fitness function is those parts of an environment that are designed to assess how good a solution fits with the problem.
As in nature, only the fittest individuals are stochastically selected to recombine or survive, based on their more deterministic fitness levels.
If only the fittest individuals were to be selected, then the search would most likely end in the first available local optima,
before the solution space has been desirably explored. After the first selection, individuals are recombined, and their offspring mutated.
Now, all the exploration steps are completed, for the current generation, and the fittest individuals of the current solution-pool are then brought to the next generation.
The evolution typically runs until certain criteria have been met, or until a set number of generations has been reached.
The best solutions can then be saved, and the rest wiped out, for a new epoch to start. \cite{evoalg_book}
The SGA has inspired mind-boggling variations of algorithms under the relatively wide EA umbrella, and among these are the
Island Model, NeuroEvolution, ..., and ... .

\subsection{The Island Model}

\subsection{NeuroEvolution}

