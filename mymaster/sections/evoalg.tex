\paragraph{Nature} is exceptional at inventing new complex, resilient solutions through the evolution of organizations, from the molecular level to the ecosystems.
This project uses biologically inspired Evolutionary Algorithms (EAs) to develop algorithms to run on hardware.
See \vref{sect:eh} for more about Evolvable Hardware.
In biology, Deoxyribonucleic acid (DNA) governs the processes and structure of the cells of living cellular organisms ).
The minimal units of DNA are called \textbf{genes}, and they are structured in \textbf{chromosomes}. They determine the physical characteristics and development of the organism.
In EAs, the chromosomes may take the form of data structures like strings, arrays, or graphs where each segment of the data structure is a gene.
Another term is \textbf{allele}. Individuals inherit two alleles for each gene, one from each parent. Alleles represent alternative forms of a gene that arise by mutation and can, for example, result in different \textbf{phenotypic} traits, like different pigmentation.
The aggregation of the alleles for a specific gene is termed the \textbf{genotype} for the gene.
Although the genotype is the most influential factor for the phenotype, other factors also play an important role, like environmental factors.
\cite{trefzer_evolvable_2015}
In EAs, there is usually only one allele per gene, which results in genotypes normally being identical to the chromosome.
The phenotype is usually the final solution of the EA.

\paragraph{EAs in general} are population-based optimization algorithms, adopting a higher-level of abstraction of an evolutionary scheme found in nature.
The classical Simple Genetic Algorithm (SGA) involves individuals, phenotypes, genotypes, fitness function, selection, recombination, and mutation.
A \textbf{gentotype} in EAs is each candidate solution, with its genes being the parameters of the solution. A \textbf{phenotype} is the resulting characteristic of a particular solution when in contact with the environment.
The fitness function is those parts of an environment that are designed to assess how good a solution fits with the problem.
As in nature, individuals are stochastically selected to recombine or survive based on their more deterministic fitness levels.
If only the fittest individuals were to be selected, then the search would most likely end prematurely in the first available local optima.
After the first selection, individuals are recombined, and their offspring mutated.
In optimization algorithms like EAs, there needs to be a balance between how much the best solutions are \textbf{exploited}, and how much the solution space is \textbf{explored} for new solutions.
The evolution typically runs until certain criteria are met or until a set number of generations has been reached.
\cite{eiben_introduction_2015}
The SGA has inspired mind-boggling variations of algorithms under the relatively wide EA umbrella, and examples of these are the
Island Model \cite{schuman_parallel_2016} and Evolutionary Optimization \cite{schuman_evolutionary_2016}.



