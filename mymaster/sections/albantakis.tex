\section{Albantakis et al., Evolution of Integrated Causal Structures in Animats Exposed to Environments of Increasing Complexity, 2014}
The study uses the animat approach,  \vref{sect:agent}, to evolve autonomous agents on a traditional, discrete-state computer.
The authors use a Simple Genetic Algorithm to evolve the animats, \vref{sect:ea}.
As is required in the animat approach, the study varies the complexity of the task environments.
Through analysis by use of the IIT, the researchers show that ``in complex environments with a premium on context-sensitivity and memory, integrated brain architectures have an evolutionary advantage over modular ones.''\cite{albantakis_evolution_2014-1}

The study finds that during evolution, some animats may develop a system that consists of one main conceptual complex. In contrast, other animats develop networks of segregated modules with feed-forward architecture. The study assumes that the evolution of animats will follow the same patterns as a natural evolution, where characteristics of economy, functions, adaptability, and robustness are favorable to the simplicity of design.

Each neural network "animat" consists of eight deterministic logic-gate elements. These are two sensors, four hidden elements, and two motors (left, right).
For every animat, a genotype defines the network architecture and logic-functions for each node.

The animats are evolved over 60 000 generations, starting with an initial population of 100 animats with no connections between the hidden elements.
For mutation, 100 animats are selected using roulette wheel selection, see sect. \vref{sect:ea}.
The reproduction of animats happens without crossover, and up to three different mutation mechanics can happen, at a probabilistic rate.\cite{albantakis_evolution_2014}

The three different mechanics are: Point mutations, by a probability of $p = 0.5 \%$ per loci, with uniform integer replacement. Deletion, by $p = 2\%$ per genotype, a sequence between 16 and 512 adjacent loci is deleted. Duplication, by $p = 5\%$ per genotype, a sequence between 16 and 512 adjacent loci is duplicated and inserted at a random location within the genotype.
The genotype is always between 1000 and 20 000 loci. Sequences of 10 loci encode one set of connections.\cite{albantakis_evolution_2014}

The environment is a "falling blocks game," with varying shape and size of the blocks. Additionally, the task for each type of block varies between catching and avoiding the block.

Most importantly, the work of Albantakis et al. applies the IIT and makes three predictions:
\begin{itemize}
\item{}
The number of concepts, and their $\phi^max$ (see sect. \vref{sect:iit}), should increase during adaption, proportional to the amount of internal computation necessary to solve a task.
\item{}
Given the limited number of hidden elements, integration should also increase during adaption, particularly in tasks that require more memory.
\item{}
When the animat has a limited sensory and motor capacity, it needs to rely more on memory when the complexity of the environment increases. Therefore the integration of the animat should also increase during evolution, when under sensory or motor limitations.\cite{albantakis_evolution_2014}
\end{itemize}

The study does find that during evolution, complex environments tend to lead to an increased number of concepts (functions) in the causal structure and also more integrated conceptual structures (see sect. \vref{sect:iit}).
Animats with more integrated structures also seem to be more robust, which makes sense because when there is a limited amount of elements in the brain, more integrated structures can contain more functions.
The study also found that animats with higher fitness usually had a higher degeneracy. In biology, degeneracy means that structurally different elements can perform the same function and correlates with flexibility and robustness. The author states that this could also lead to the development of higher-order functions.\cite{albantakis_evolution_2014}

What does the study say of the IIT, and how can it be used? According to IIT, integrated conceptual structures underlie consciousness \cite{tononi_integrated_2016}. This study found that animats with a higher number of integrated conceptual structures found an advantage in complex environments, which may indicate that the IIT can be a useful tool in the quest for intelligent machines.





